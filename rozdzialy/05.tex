\section{12.11.2025}{Zespolone formy rózniczkowe}

% TODO: symbol \Omega^{q,p}

Zadanie z listy: $H^1(\Sigma)\ni[\alpha]\neq0\implies\exists$ pętla $\gamma$ taka, że $I_\gamma([\alpha])=\int_\gamma\alpha\neq0$

\begin{theorem}{}{}
  $\Sigma$ - zwarta powierzchnia bez brzegu

  $$H^1(\Sigma)\times H^1(\Sigma)\ni([\theta],[\alpha])\mapsto\int_\Sigma\theta\wedge\alpha\in\R$$
  jest formą $2$-liniową, antysymetryczną, niezdegenerowaną ($\implies 2|\dim(H^1(\Sigma))$)
\end{theorem}

Pytanie: czy $S^1\cong T^2$? Gdyby tak było, to istniałby dyfeomorfizm \( f: S^1 \to T^2 \), który indukowałby izmorofizm \( f^\star : H^1 (S^1) \to H^1 (T^2) \). Ewidentnie tak nie jest, bo $H^1(S^1)=0$, a $H^1(T^2)=\Z^2$.

\subsection{Zespolone formy różniczkowe}

Niech $\Sigma$ będzie powierzchnią Riemanna. Potrzebujemy zespolonej przestrzeni kostycznej/zespolone $1$-formy. Definiujemy

$T_p^*\Sigma^\C = \Hom_\R (T_p\Sigma,\C )$

$f \in C^\infty(\Sigma, \C )\to (df)_p \in T_p^*\Sigma$

np. lokalnie $z=x+iy$ $d(x+iy)=dx_idy:=dz$

$\Omega^1_\C(\Sigma)$=gładkie cięcia $T^*\Sigma^\C=\bigcup_{p\in\Sigma}T_p^*\Sigma$

sprzężenie $\omega\in\Omega^1_\C$ to $\overline{\omega}\in\Omega^1_\C$ t,że $\overline{\omega}(p)=\overline{\omega(p)}$

\begin{example}
  $\omega=a+bi$, $a,b\in\Omega_\R^1(\Sigma$, $\overline{\omega}=a-ib$
\end{example}

\begin{definition}{}{}
  Zespolona struktura na rzeczywistej p.liniowej $V$ jest zadana przez $\R$-liniowe $J:V\to V$ takie, że $J^2=-Id$. Definiujemy wtedy mnożenie przez liczby zespolone jako
\[ 
    (a + bi)v := a + b(Jv). 
\]
\end{definition}

\begin{fact}{}{}
  Na $T_p\Sigma$ istnieje jedyna struktura zespolona $J$ taka, że gładka $f:\Sigma\to \C$ jest holomorficzna w $p$ $\iff$ $df_p:(T_p\Sigma,J)\to\C$ jest $\C$-liniowe

  w mapie $J[\alpha]_p=[p+i(\gamma-p)]_p$
\end{fact}

\begin{proof}
  ćwiczenie
\end{proof}

\begin{fact}{}{}
  Niech $(V, J)$ będzie jak wyżej oraz niech $A:V\to\C$ będzie $\R$-liniowe. Wtedy $A$ zapisuje się jednoznacznie jako suma $\C$-liniowego $A'$ i $\C$-antyliniowego $A''$, tzn. \( A' \) i \( A'' \) t. że

\begin{align*}
    A'(Jv)  &= iA'(v) \\
    A''(Jv) &= -iA''(v)
\end{align*}

\end{fact}

\begin{proof}
    Określamy potrzebne odwzorowania jako:
    \begin{align*}
        A'(v) &=\frac{1}{2}(A(v)-iA(Jv)) \\
        A''(v) &=\frac{1}{2}(A(v)+iA(Jv))
    \end{align*}
\end{proof}

Rozkład z powyższego faktu daje rozkłady 
\begin{itemize}
  \item $T_p^*\Sigma^\C = T_p^*\Sigma'\oplus T_p^*\Sigma''$
  \item analogicznie $\Omega_\C^1$
  \item rozkład $d$:
    \begin{center}
      \begin{tikzcd}
        \Omega^{0,1} \arrow[r, "\partial"] & \Omega_\C^2\\ 
        \Omega_\C^0\arrow[u, "\overline{\partial}"]\arrow[r, "\partial"]&\Omega^{1,0}\arrow[u, "\overline{\partial}"]
      \end{tikzcd}
    \end{center}
  \item we współrzędnych $z=x+iy$
  
    $dz=dx+idy$ jest $\C$-liniowe

    $d\overline{z}=dx-idy$ jest $\C$-anytliniowe

    $\alpha dz\in\Omega^{1,0}$, $\beta d\overline{z}\in\Omega^{0,1}$, gdzie $\alpha,\beta:\Sigma\to\C$ są gładkie, niekoniecznie $0$

    dla $f:\Sigma\to\C$ ($C^\infty$, niekoniecznie $O$)
    
    $df=f_xdx+dydy=f_x\frac{dz+d\overline{z}}{2}+f_y\frac{dz-d\overline{z}}{2}=\underline{\frac{1}{2} (fx-ify)dz}_{\partial f}+\underline{\frac{1}{2}(...)}_{\overline{\partial}}$
\end{itemize}

Równania Cauchy'ego-Riemanna $\overline{\partial}f=0$ - wtedy $df=\partial f=f'(z)dz$

$d:\Omega^1\to\Omega^2$ są dwie możliwości
\begin{itemize}
  \item $d(Ad\overline{z})=\frac{\partial A}{\partial z}dzd\overline{z}=\partial (Adz)$

    $d(\alpha dz)=\frac{\partial \alpha}{\partial z}dzd\overline{z}=\overline{\partial}(\alpha dz)$
    i analogicnzie dla $\beta d\overline{z}$
\end{itemize}

\begin{definition}{}{}
  Forma typu $(1,0)$ jest holomorficzna jeśli jej $\overline{\partial}=0$ (jeśli jej $d=0$) (lokalnie $\alpha dz$ z holomorficzną $\alpha$)
\end{definition}

Jeśli $S\subseteq \Sigma$ jest podpowierzchnią zwartą z brzegiem, $\alpha$ jest holo $1$-formą to $\int_{\partial S}\alpha=0$ (Stokes/Cauchy) (dla holo $\alpha$ $d\alpha=0$)


Laplasjan

$\Delta=2i\overline{\partial}\partial:\Omega^0\to\Omega^2$
$\Delta f=2i\frac{1}{2}(\partial_x+i\partial_y)\frac{1}{2})\partial_x-i\partial_y)fd\overline{z}dz=-(\partial_x^2+\partial_y^2)fdxdy$

\begin{definition}{}{}
  Funkcję gładką $f$ nazwiemy harmoniczną, gdy $\Delta f=0$.
\end{definition}

\begin{fact}{}{}
    \begin{enumerate}[label=\alpha]
        \item Jeśli funkcja \( f: \Sigma \to \C \) jest holomorficzna, to \( \Re f \) oraz \( Im f \) są funkcjami harmonicznymi. 
        \item Jeśli \( \varphi: \Sigma \to \R \) jest harmoniczna, to \textbf{lokalnie} mamy \( \varphi = \Re f \) dla holomorficznej funkcji \( f \).
    \end{enumerate}
\end{fact}

\begin{proof}
  $A:i\overline{\partial}\phi+\overline{i\overline{\partial}\phi}\in\Omega_\R^1$

  $0=\overline{\partial}\partial\phi\implies dA=0$

  o jezunku sie jasiu rozpisal
\end{proof}

zasada maksimum - harmoniczna nie osiaga max w żadnym punkcie dziedziny czy cos

kohomologie dolbeault
obrazki

w jaki sposób się kohomologie przydają żeby konstruować holomorficzne funkcje

\begin{fact}{}{}
  Na zwartej powierzchni Riemanna istnieje niestała funkcja meromorficzna
\end{fact}

strategia:

zaczynamy na dyszczku i próbujemy robić kontynuacje, obrazek

dobieramy $\beta\in\Omega^0_c(U)$ która jest równa $1$ w pobliżu zera 

jak mi sie nie chce

